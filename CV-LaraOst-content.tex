\usepackage{csquotes}
\usepackage[main=english,ngerman]{babel}
\usepackage[maxbibnames=99,
            sorting=ydnt,
            block=ragged,
            backend=biber,
            doi=true,
            url=false]{biblatex}
\usepackage[hidelinks]{hyperref}
\usepackage{fontawesome}
\usepackage{datetime2}

\RequirePackage{fontspec}
\setmainfont{Source Serif Pro}
\setsansfont{Source Sans Pro}

% Define \isEnglish to use English, define \isGerman to use German.
% Using both at the same time is not allowed.
%
% Requires \usepackage{datetime2} to define a \monthname that doesn't screw up in combination with the otherlanguage environment.

\newbool{langde}
\newbool{langen}

\newcommand{\monthname}[1]{#1}

\ifdefined\isEnglish
    \setbool{langen}{true}
    \renewcommand{\monthname}[1]{\DTMenglishmonthname{#1}}
    \ifdefined\isGerman
        \errmessage{Can't have both English and German}
    \fi
\fi
\ifdefined\isGerman
    \renewcommand{\monthname}[1]{\DTMgermanmonthname{#1}}
    \setboolean{langde}{true}
\fi

\newcommand{\en}[1]{%
    \iflangen#1\fi
}
\newcommand{\de}[1]{%
    \iflangde%
        \begin{otherlanguage}{ngerman}%
            #1
        \end{otherlanguage}%
    \fi%
}

\de{\title{Lebenslauf}}
\en{\title{Resume}}
\beforename{Dr.}
\author{Lara Chiara Ost}
\aftername{MSc ETH MSc}
\de{
    \date{\the\day.~\monthname{\month} \the\year}
}
\en{
    \date{\monthname{\month} \the\day, \the\year}
}

\de{\addbibresource{publications-de.bib}}
\en{\addbibresource{publications.bib}}
\nocite{*}

\newbibmacro{string+doi}[1]{%
  \iffieldundef{doi}{\iffieldundef{url}{#1}{\href{\thefield{url}}{#1}}}{\href{http://dx.doi.org/\thefield{doi}}{#1}}}	
\DeclareFieldFormat*{title}{\usebibmacro{string+doi}{\mkbibemph{#1}}}
\newcommand{\nohttpurl}[1]{\href{http://#1}{\nolinkurl{#1}}}

\begin{document}

\maketitle

\begin{linkblock}{3}
\icontext{linkedin}{
    \href{www.linkedin.com/in/lara-ost-032463136}{lara-ost-032463136}}
\icontext{github}{\href{https://github.com/laraost}{laraost}}
\end{linkblock}

\de{\section*{Erfahrung}}
\en{\section*{Experience}}

\de{
\bigcventry{\fromto{Oktober 2019}{März 2025}}
{Prae-Doc Assistentin in der Forschungsgruppe Theory and Applications of Algorithms, Universität Wien}
{
    \begin{itemize}
        \item Forschung zu dynamischen Algorithmen für Graphprobleme, topologische Datenanalyse und Differential Privacy
        \item Implementierung und experimentelle Evaluierung von Algorithmen
        \item Teil des Organisations-Teams des
            \enquote{Symposium on Experimental Algorithms} 2024
        \item Unterricht der Kurse
            \enquote{Algorithms and Data Structures 2} und
            \enquote{Advanced Algorithms};
            Mitbetreuung von Bachelor- und Masterarbeiten
    \end{itemize}
}
}
\en{
\bigcventry{\fromto{October 2019}{March 2025}}
{Prae-Doc Assistant in the research group Theory and Applications of Algorithms, University of Vienna}
{
    \begin{itemize}
        \item Researched dynamic algorithms for graph problems,
            topological data analysis, differential privacy and trajectory analysis
        \item Developed, optimized and benchmarked performance oriented implementations of algorithms in C++
        \item Evaluated reproducibility of published results as part of the ALENEX 2025 Artifact Evaluation Committee;
            acted as (sub-)reviewer for conferences (ALENEX, SODA, ESA, SEA, ACDA, SIROCCO)
        \item Local organizer of the
            \enquote{Symposium on Experimental Algorithms} 2024
        \item Taught courses
            \enquote{Algorithms and Data Structures 2} and
            \enquote{Advanced Algorithms};
            co-supervised a bachelor's thesis and master's theses
    \end{itemize}
}
}

\de{
\cventry{\fromto{Oktober 2018}{Juni 2019}}{Tutorin für \enquote{Foundations of Computer Graphics}, Universität Wien}
}
\en{
\cventry{\fromto{October 2018}{June 2019}}{Teaching assistant for \enquote{Foundations of Computer Graphics}, University of Vienna}
}



\de{\section*{Projekte}}
\en{\section*{Projects}}

\de{
\projectblock{BananaPersist}{2025}{%
    \nohttpurl{github.com/laraost/BananaPersist},
    beschrieben in \cite{ost24.1,cultrera24}
}{
    Datenstruktur zur effizienten Berechnung des \enquote{Persistence Diagrams} von Zeitreihen.
    Dies ist die erste Datenstruktur für das Problem, die Änderungen am Input zulässt und dabei das Diagramm effizient aktualisiert.
    Ich habe die Datenstruktur mit-designed und anschließend in C++ implementiert.
}
}
\en{
\projectblock{BananaPersist}{2025}{%
    \nohttpurl{github.com/laraost/BananaPersist},
    described in \cite{ost24.1,cultrera24}
}{
    Data structure to maintain the persistence diagram of time series under changes to the input.
    %This is the first data structure for this problem.
    This work introduces the novel banana tree structure, which can be efficiently updated when the input is modified. 
    I co-designed the data structure, implemented it in C++ and performed experiments to analyze its performance.
}
}

\de{
\projectblock{DyDJ-Match}{2023}{%
    \nohttpurl{github.com/DJ-Match/DyDJ-Match},
    beschrieben in \cite{hanauer22}  
}
{
    Dynamische Algorithmen zur Berechnung von maximalen disjunkten Matchings in gewichteten Graphen.
    Das Problem tritt in der Konfiguration von rekonfigurierbaren optischen Netzwerken auf.
    Ich habe die Algorithmen mit-entworfen, in C++ implementiert und deren Performance evaluiert.
}
}
\en{
\projectblock{DyDJ-Match}{2023}{%
    \nohttpurl{github.com/DJ-Match/DyDJ-Match},
    described in \cite{hanauer22}  
}
{
    Dynamic algorithms to compute maximum disjoint matchings in weighted graphs.
    We are the first to study this problem in the dynamic setting, where inputs may change. 
    This work is motivated by an application to optical networks in data centers.
    I co-designed the algorithms, implemented them in C++ and evaluated their performance.
}
}

\de{
\projectblock{Beitrag zu KaHIP}{2020}{%
    \nohttpurl{kahip.github.io}, beschrieben in \cite{ost21}
}{
    Algorithmus zur Berechnung von Knotenordnungen mit kleinem \enquote{Fill-In}, basierend auf Nested Dissection.
    Ich habe den Algorithmus in C++ implementiert, optimiert, experimentell analysiert und in das bestehende KaHIP-Framework (Version 3.0) integriert.
}
}
\en{
\projectblock{Contribution to KaHIP}{2020}{%
    \nohttpurl{kahip.github.io}, described in \cite{ost21}
}{
    I contributed an algorithm to compute node orderings with small fill-in, based on nested dissection.
    This work incorporates reduction rules into nested dissection to improve the algorithm's performance in practice.
    I implemented the algorithm in C++, analyzed it in extensive experiments and integrated it into the existing KaHIP-framework (version 3.0).
}
}

\de{\section*{Ausbildung}}
\en{\section*{Education}}

\de{
\bigcventry{\fromto{März 2020}{Februar 2025}}
{Dr. techn. in Informatik, Universität Wien}
{
    Mitglied der \enquote{Vienna Graduate School on Computational Optimization} \\
    Dissertation: \emph{\enquote{Engineering Efficient Algorithms for the Analysis of Dynamic Data}} \\
    Betreuerinnen: Prof.~Monika Henzinger, Prof.~Kathrin Hanauer
}
}
\en{
\bigcventry{\fromto{March 2020}{February 2025}}
{Dr. techn. in Computer Science, University of Vienna}
{
    Member of the \enquote{Vienna Graduate School on Computational Optimization} \\
    Thesis: \emph{\enquote{Engineering Efficient Algorithms for the Analysis of Dynamic Data}} \\
    Supervisors: Prof.~Monika Henzinger, Prof.~Kathrin Hanauer
}
}

\de{
\cventry{\fromto{März}{Juni 2023}}{
    Forschungsaufenthalt, DTU Kopenhagen\\
    Forschung zu Algorithmen für Subtrajectory Clustering
}
}
\en{
\cventry{\fromto{March}{June 2023}}{
    Research stay, DTU Kopenhagen \\
    Research on discrete subtrajectory clustering algorithms
}
}

\de{
\bigcventry{\fromto{Oktober 2017}{September 2019}}
{MSc in Computational Science, Universität Wien, mit Auszeichnung bestanden}
{
    Masterarbeit: \emph{\enquote{Reduced Nested Dissection for Fill Reducing Node Orderings} \\}
}
}
\en{
\bigcventry{\fromto{October 2017}{September 2019}}
{MSc in Computational Science, University of Vienna, pass with distinction}
{
    Master's thesis: \emph{\enquote{Reduced Nested Dissection for Fill Reducing Node Orderings} \\}
}
}

\de{
\bigcventry{\fromto{September 2015}{Mai 2016}}
{MSc ETH in Chemie, ETH Zürich}
{
    Masterarbeit: \emph{\enquote{Prejudice-Free Exploration of Reaction Space with Quantum Chemical  Methods}} \\
}
}
\en{
\bigcventry{\fromto{September 2015}{May 2016}}
{MSc ETH in Chemistry, ETH Zürich}
{
    Master's thesis: \emph{\enquote{Prejudice-Free Exploration of Reaction Space with Quantum Chemical  Methods}} \\
}
}

\de{\cventry{Herbst 2014}{Auslandssemester, Universität Uppsala}}
\en{\cventry{Fall 2014}{Exchange semester, Uppsala University}}

\de{\cventry{\fromto{September 2012}{Mai 2015}}{BSc ETH in Chemie,
    ETH Zürich}}
\en{\cventry{\fromto{September 2012}{May 2015}}{BSc ETH in Chemistry,
    ETH Zürich}}

\de{
\cventry{\fromto{Oktober 2011}{Juni 2012}}{Studium in klassischer Archäologie und Altgriechisch, Universität Tübingen}
}
\en{
\cventry{\fromto{October 2011}{June 2012}}{Studies of classical archaeology and Greek philology, University of Tübingen
}
}

\clearpage

\de{\section*{Kenntnisse}}
\en{\section*{Skills}}

\de{\cventry{Sprachen}{Deutsch, Englisch,
                       Großes Latinum, Graecum,
                       Grundkenntnisse in Französisch, Spanisch und Schwedisch}}
\en{\cventry{Languages}{German, English,
                        basics in French, Spanish and Swedish,\\
                        Großes Latinum (advanced proficiency certificate in Latin),\\
                        Graecum (proficiency certificate in Ancient Greek)}}

\de{
\cventry{Programmiersprachen}{C++, R, Python, Rust, Java, SQL, Matlab, JavaScript, LaTeX}
}
\en{
\cventry{Programming Languages}{C++, R, Python, Rust, Java, SQL, Matlab, JavaScript, LaTeX}
}

\de{
\cventry{Technologien}{Linux, vim, git, cmake, meson}
}
\en{
\cventry{Technologies}{Linux, vim, git, cmake, meson}
}

\de{\section*{Andere Aktivitäten}}
\en{\section*{Other Activities}}

\de{
\cventry{September 2018}{Besuch der Summer School Deep Learning and Visual Data Analysis, Universität Wien}
}
\en{
\cventry{September 2018}{Summer School Deep Learning and Visual Data Analysis,\\ University of Vienna}
}

\de{
\cventry{2011}{Qualifikation als nebenberufliche Kirchenmusikerin mit D-Prüfung, \\
    Evangelische Landeskirche Baden, Deutschland}
}
\en{
\cventry{2011}{Qualification as part-time church musician with D-examination, \\
    Protestant Church of Baden, Germany}
}

\de{\printbibliography[title=Publikationen]}
\en{\printbibliography[title=Publications]}

\label{lastpage}~

\end{document}
